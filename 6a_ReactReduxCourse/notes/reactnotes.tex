\documentclass[8pt,a4paper]{extarticle}
%\documentclass[8pt,twocolumn,a4paper]{extarticle}
%\documentclass[6pt,a4paper]{article}
\usepackage[utf8]{inputenc}
\usepackage{amsmath}
\usepackage{amsfonts}
\usepackage{amssymb}
\usepackage{graphicx}
\usepackage{enumitem}
\usepackage{setspace} 
\usepackage{float}
\usepackage{url}
\usepackage[usenames,dvipsnames,svgnames,table]{xcolor}
\usepackage{multicol}

\pagecolor{black}
\setlist{nolistsep}
\topmargin -2cm
\oddsidemargin -1.5cm
\evensidemargin -1.5cm
\textwidth 18.75cm
\textheight 26cm
\linespread{1.00}
\color{YellowOrange}

%Other Colors: ForestGreen,Cerulean,Salmon,Purple, Red, White

\twocolumn
\title{Title Here}
\author{Author Name Here}

\begin{document}

\newcommand{\mitem}{\item[$\square$]}
\newcommand{\mmitem}{\item[$\triangledown$]}

\maketitle


\section*{First React Application:}

\begin{itemize}
\item To start a React Project, most developers will utilize the create-react-app project template (a node package). Simply install this package and run it to start.
Specifically:

\begin{verse}

npx create-react-app <project name>
mkdir <project name>
npm start

\end{verse}

to install and get the local server running.

\item \textbf{Project Structure:}
\item \textbf{./node_modules:} All locally installed node dependences are stored here.
\item \textbf{./public:} index.html (front page, robots.txt and favicons stored here.
\item \textbf{./src:} Contains the React Apps (self-contained pieces of interface+code), and support files, including:

- App.js (contains template React App)
- .test files (to run tests later)
- index.js: Where we Insert our App DOM object into the index.html DOM tree (at root).

\item A Most Minimal Example: There are a lot of files in the src folder, but they are not all necessary for react to run. The only necessary file is the index.js file. The following minimial code is needed to insert our content into the index.html DOM tree:

\begin{verbatim}
import React from 'react';
import ReactDOM from "react-dom";
ReactDOM.render(<div>Our React Element </div>, 
document.getElementById("root"));
\end{verbatim}

\item In react, we typically bundle reusable UI elements into \textbf{Components}. 
\item A react app will contain many different \textbf{Components}.
\item Naturally, react has a Component class that can be imported from the react core library.

\begin{verbatim}
import { Component } from 'react';
\end{verbatim}

\item Note that Classes, extended Sub-Classes and object Instances are used heavily in React - so be familiar with OOP.
\item A typical design pattern in React will be to extend template component classes, and add additional fields/methods to implement your application.
\item If you write javascript in the App.js and index.js - console.log() will output directly to a browser inspector.
\item React automatically detects saved changes, and the local server updates. So you can update accordingly.
\item
\item
\item
\item
\end{itemize}

\section{}

\subsection{}

\subsubsection{}

\begin{itemize}
\item
\item
\item
\item
\item
\item
\item
\item
\item
\item
\item
\item
\item
\item
\item
\item
\item
\end{itemize}

\section{}

\subsection{}

\subsubsection{}

\begin{itemize}
\item
\item
\item
\item
\item
\item
\item
\item
\item
\item
\item
\item
\item
\item
\item
\item
\item
\end{itemize}

\section{}

\subsection{}

\subsubsection{}

\begin{itemize}
\item
\item
\item
\item
\item
\item
\item
\item
\item
\item
\item
\item
\item
\item
\item
\item
\item
\end{itemize}

\section{}

\subsection{}

\subsubsection{}

\begin{itemize}
\item
\item
\item
\item
\item
\item
\item
\item
\item
\item
\item
\item
\item
\item
\item
\item
\item
\end{itemize}

\section{}

\subsection{}

\subsubsection{}

\begin{itemize}
\item
\item
\item
\item
\item
\item
\item
\item
\item
\item
\item
\item
\item
\item
\item
\item
\item
\end{itemize}

\section{}

\subsection{}

\subsubsection{}

\begin{itemize}
\item
\item
\item
\item
\item
\item
\item
\item
\item
\item
\item
\item
\item
\item
\item
\item
\item
\end{itemize}

\section{}

\subsection{}

\subsubsection{}

\begin{itemize}
\item
\item
\item
\item
\item
\item
\item
\item
\item
\item
\item
\item
\item
\item
\item
\item
\item
\end{itemize}

\section{}

\subsection{}

\subsubsection{}

\begin{itemize}
\item
\item
\item
\item
\item
\item
\item
\item
\item
\item
\item
\item
\item
\item
\item
\item
\item
\end{itemize}

\section{}

\subsection{}

\subsubsection{}

\begin{itemize}
\item
\item
\item
\item
\item
\item
\item
\item
\item
\item
\item
\item
\item
\item
\item
\item
\item
\end{itemize}

\section{}

\subsection{}

\subsubsection{}

\begin{itemize}
\item
\item
\item
\item
\item
\item
\item
\item
\item
\item
\item
\item
\item
\item
\item
\item
\item
\end{itemize}

\section{}

\subsection{}

\subsubsection{}

\begin{itemize}
\item
\item
\item
\item
\item
\item
\item
\item
\item
\item
\item
\item
\item
\item
\item
\item
\item
\end{itemize}

\section{}

\subsection{}

\subsubsection{}

\begin{itemize}
\item
\item
\item
\item
\item
\item
\item
\item
\item
\item
\item
\item
\item
\item
\item
\item
\item
\end{itemize}

\section{}

\subsection{}

\subsubsection{}

\begin{itemize}
\item
\item
\item
\item
\item
\item
\item
\item
\item
\item
\item
\item
\item
\item
\item
\item
\item
\end{itemize}

\section{}

\subsection{}

\subsubsection{}

\begin{itemize}
\item
\item
\item
\item
\item
\item
\item
\item
\item
\item
\item
\item
\item
\item
\item
\item
\item
\end{itemize}


\begin{thebibliography}{9}

\bibitem{ganacheSE}
\url{https://ethereum.stackexchange.com/questions/109847/how-to-install-ganache-ui-on-ubuntu-20-04-lts}
\bibitem{}
\url{}
\bibitem{}
\url{}
\bibitem{}
\url{}
\bibitem{}
\url{}
\bibitem{}
\url{}

	
\end{thebibliography}


\end{document}
